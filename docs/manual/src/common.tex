\begin{titlepage}
  \begin{center}

  {\Huge SIPO}

  \vspace{25mm}

  \includegraphics[width=0.90\textwidth,height=\textheight,keepaspectratio]{img/AFRL.png}

  \vspace{25mm}

  \today

  \vspace{15mm}

  {\Large Jay Convertino}

  \end{center}
\end{titlepage}

\tableofcontents

\newpage

\section{Usage}

\subsection{Introduction}

\par
This core converts the serial data to parallel data. This is useful for SPI and other serial data protocols. All data actions
in the core are done on a rising edge clock. The enable is used to change the rate based upon the input clock.

\subsection{Dependencies}

\par
The following are the dependencies of the cores.

\begin{itemize}
  \item fusesoc 2.X
  \item iverilog (simulation)
  \item cocotb (simulation)
\end{itemize}

\input{src/fusesoc/depend_fusesoc_info.tex}

\subsection{In a Project}
\par
This core is made to interface for serial to parallel data. On each enable pulse on a positive clock edge the input serial
line will be sampled. The counter starts at 0 and the shift is complete once it hits BUS\_WIDTH * 8. For a 32 bit word you will
need 32 enable pulses. These will correspond to counter output 0 to 31. 32 will signal the shift is done and the parallel output
data is valid. Once it is valid data can be read and load asserted. Load will reset the core and clear it for the next shift operation.
\begin{enumerate}
  \item Set ena to 1 at rising clock edge for one clock cycle
  \item Continue till the counter hits the number of bits needed.
  \item Read the parallel data output to capture data.
  \item Assert load to 1 to restart counter and clear parallel register.
\end{enumerate}

\begin{lstlisting}[language=Verilog]
\end{lstlisting}

\section{Architecture}
\par
The only module is the sipo module. It is listed below.

\begin{itemize}
  \item \textbf{sipo} Convert serial data to parallel by shifting in data on each enable pulse on a rising clock. (see core for documentation).
\end{itemize}

Please see \ref{Module Documentation} for more information.

\section{Building}

\par
The SIPO core is written in Verilog 2001. They should synthesize in any modern FPGA software. The core comes as a fusesoc packaged core and can be
included in any other core. Be sure to make sure you have meet the dependencies listed in the previous section.

\subsection{fusesoc}
\par
Fusesoc is a system for building FPGA software without relying on the internal project management of the tool. Avoiding vendor lock in to Vivado or Quartus.
These cores, when included in a project, can be easily integrated and targets created based upon the end developer needs. The core by itself is not a part of
a system and should be integrated into a fusesoc based system. Simulations are setup to use fusesoc and are a part of its targets.

\subsection{Source Files}

\input{src/fusesoc/files_fusesoc_info.tex}

\subsection{Targets}

\input{src/fusesoc/targets_fusesoc_info.tex}

\subsection{Directory Guide}

\par
Below highlights important folders from the root of the directory.

\begin{enumerate}
  \item \textbf{docs} Contains all documentation related to this project.
    \begin{itemize}
      \item \textbf{manual} Contains user manual and github page that are generated from the latex sources.
    \end{itemize}
  \item \textbf{src} Contains source files for the core
  \item \textbf{tb} Contains test bench files for iverilog and cocotb
    \begin{itemize}
      \item \textbf{cocotb} testbench files
    \end{itemize}
\end{enumerate}

\newpage

\section{Simulation}
\par
There are a few different simulations that can be run for this core.

\subsection{iverilog}
\par
iverilog is used for simple test benches for quick verification, visually, of the core.

\subsection{cocotb}
\par
To use the cocotb tests you must install the following python libraries.
\begin{lstlisting}[language=bash]
  $ pip install cocotb
\end{lstlisting}

\begin{itemize}
  \item \textbf{sim\_cocotb} Standard simulation of SIPO conversion.
\end{itemize}

Then you must use the cocotb sim target. The targets above can be run with various parameters.
\begin{lstlisting}[language=bash]
  $ fusesoc run --target sim_cocotb AFRL:simple:sipo:1.0.0
\end{lstlisting}

\newpage

\section{Module Documentation} \label{Module Documentation}

\par

\begin{itemize}
\item \textbf{sipo} SIPO converter\\
\item \textbf{tb\_sipo-v} Verilog test bench\\
\item \textbf{tb\_cocotb-py} Cocotb python test routines\\
\item \textbf{tb\_cocotb-v} Cocotb verilog test bench\\
\end{itemize}
The next sections document the module in detail.

